\chapter{$1s$原初高斯函数的积分计算方法}
\label{appendix:a}


大多数分子计算使用固定的分子坐标系,使得基函数以该坐标系中的位置矢量
$\bo{R}_A$为中心,如\autoref{figA.1}所示。一个以$\bo{R}_A$为中心的
位置矢量$\bo{r}$的值将会依赖于$\bo{r}-\bo{R}_A$,因此我们可以写一个一般化的基函数
$\phi_\mu(\bo{r}-\bo{R}_A)$来代表它是以$\bo{R}_A$为中心的。在一个分子计算的过程中
我们需要计算数量庞大的包含不同中心$\phi_\mu(\bo{r}-\bo{R}_A)$的单电子和双电子积分。
如果我们使用的基函数包括四个或者更多的中心,那么我们的双电子积分将会包含$1-$、$2-$、$3-$和
$4-$中心积分。核吸引势积分最多只能处理到三中心。这些多中心(超过$2$)的积分对于Slater型函数
来说非常难以处理,但是对于Gaussian型函数来说处理起来相对简单。因此许多多原子计算使用Gaussian函数。
\begin{figure}[h]
	\begin{tikzpicture}[scale=2,inner sep=0,arrows=-latex]
		\draw (0,0)--(0,2);
		\draw (0,0)--(-1.2,-1.2);
		\draw (0,0)--(2,0);
		\node[shape=circle,draw,minimum size=.4cm] (j) at (2,1) {\bf j};
		\node[shape=circle,draw,minimum size=.4cm] (B) at (-1.2,0.4) {\bf B};
		\node[shape=circle,draw,minimum size=.4cm,] (i) at (.7,1.8) {\bf i};
		\node[shape=circle,draw,minimum size=.4cm] (A) at (-.9,1) {\bf A};
		\draw (0,0)--node[right=13pt]{$\mathbf{r_j}$}(j);
		\draw (0,0)--node[right=5pt]{$\mathbf{r_i}$}(i);
		\draw (0,0)--node[right=2pt,above=7pt]{$\mathbf{R_A}$}(A);
		\draw (0,0)--node[left=3pt,below=2pt]{$\mathbf{R_B}$}(B);
		\draw (B)--node[left=.2cm,]{$\mathbf{R_{AB}}=\mathbf{R_A-R_B}$}(A);
		\draw (A)--node[left=.2cm]{$\mathbf{r_{iA}}=\mathbf{r_i-R_A}$}(i);
		\draw (A)--node[pos=.7,above]{$\mathbf{r_{jA}}=\mathbf{r_j-R_A}$}(j);
		\draw (j)--node[right=.2cm]{$\mathbf{r_{ij}}=\mathbf{r_i-r_j}$}(i);
		\draw[draw=none] (.52,-.5)node{$i,j \equiv \text{电子}$};
		\draw[draw=none] (.4,-.7)node{$A,B\equiv\text{核}$};
	\end{tikzpicture}
	\caption{分子坐标系}
	\label{figA.1}
\end{figure}

% \newpage
在高斯函数的计算中,收缩型的高斯函数
