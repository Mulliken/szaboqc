\chapter{几何结构优化与解析导数法(作者:M.C.Zerner)}
\section{简介}

分子构象的再现和预测是分子量子力学的最成功的应用之一。
许多情况下,对于很多简单的分子轨道模型或者最小基组的从头算方法,键长可以重现到$\pm 0.02$\r A,
键角可以重现到$5^{\circ}$。
对于大一点的基组,尤其是那些双$\zeta$加极化的基组;和包含电子相关能的方法,现在生成的几何构型已经
可以达到晶体学精度了。
随着构象计算的日益成功,人们甚至可以选择孤立分子的计算结果,而不是在凝聚介质中获得的实验结果,因为前者可能更适合气相。

除了产生关于势能面全局最小点的信息外,量子力学计算还可以产生局域最小点的信息,这些局域最小点可能不会被直接观测到,但是
很可能会被包哦旱灾反应路径中。类似的,关于过渡态和能垒的信息也可以得到,这些信息一般很难甚至不可能通过其他方式获得。

收集一个势能面上这些所有的信息是很困难的。对于N个原子的体系,它的能量是一个具有$3N-6$(或者$3N-5$)个自由度的函数。
为了进行详细的统计计算,人们可能不得不面对这个“3N”问题,并访问势能面上所有热力学可得的区域。
然而,本附录只涉及势能面的一小部分:那些对应于极小值的点,代表稳态或亚稳态的构象,以及对应于过渡态的点。
%%%%%%%
\section{总则}

在Born-Oppenheimer近似下得到的分子体系的能量$E$是一个以核坐标为参数的函数,记核坐标为
$\mathbf{X}^{\dagger}=(X_1,X_2,\dots,X_{3N})$。
我们希望从$E(\mathbf{X})$进而得到$E(\mathbf{X_1})$,$\mathbf{q=(X_1-X)}$。
我们将能量对$\mathbf{X}$进行泰勒展开:
\begin{align}
	\label{C.1}
	E(\mathbf{X_1})=E(\mathbf{X})+\mathbf{q}^{\dagger}\mathbf{f(X)}
                    +\frac{1}{2}\mathbf{q}^{\dagger}\mathbf{H(X)}\mathbf{q}+\dots
\end{align}
式中梯度为
\begin{align}
	f_i=\frac{\partial E(\mathbf{X})}{\partial X_i}
    \nonumber
\end{align}
Hessian矩阵元为
\begin{align}
	H_{ij}=\frac{\partial E(\mathbf{X})}{\partial X_i\partial X_j}
    \nonumber
\end{align}
注意,列矩阵的下标表示不同的矩阵,如$\mathbf{X_1}$、$\mathbf{X_2}$等等;而$X_i$表示矩阵$\mathbf{X}$第i个元素。
虽然泰勒展开是无穷项的,但是接近极值时,我们希望二次项是足够的;例如,对于$\mathbf{X}=\mathbf{X_e}$,其中$\mathbf{X_e}$
表示一个驻点,根据定义$\mathbf{f(X_e)=0}$,则
\begin{align}
	\nonumber
	E(\mathbf{X_1})=E(\mathbf{X_e})+\frac{1}{2}\mathbf{q}^{\dagger}\mathbf{H(X_e)}\mathbf{q}
\end{align}
类似的,
\begin{align}
	\label{C.2}
	\mathbf{f(X_1)}=\mathbf{f(X)}+\mathbf{H(X)}\mathbf{q}
\end{align}
对点$\mathbf{X_1}=\mathbf{X_e}$
\begin{align}
	\label{C.3}
	\mathbf{f(X)}=-\mathbf{H(X)}\mathbf{q}
\end{align}

\autoref{C.3}的解是不显含$\mathbf{X}$的$E(\mathbf{X})$泛函寻找多变量函数极值的最高效方法的初始点。